\documentclass[12pt]{article}
\usepackage {setspace}
\singlespace
\pagestyle{plain}
%\renewcommand{\langinfo}[1]
\usepackage{fontspec}% Enable KEYMAN functionality
\setmainfont{Libertinus Serif}
\usepackage[spanish]{babel}
%\usepackage{indentfirst}
%\setlength{\parindent}{1cm}
\usepackage{authblk}
\usepackage{avant}
\usepackage[width=160mm,top=20mm,bottom=20mm]{geometry}
\usepackage{booktabs}
\usepackage{capt-of}
\usepackage[labelfont=bf]{caption}
\usepackage{graphicx}
\usepackage{media9}
\usepackage{natbib}
\setcitestyle{notesep={:}}
\usepackage{natbib}
\usepackage{bibentry}
\usepackage{hyperref}%If colors are needed, I have to set up. See this discussion a hypersetuphttps://tex.stackexchange.com/questions/823/remove-ugly-borders-around-clickable-cross-references-and-hyperlinks?noredirect=1&lq=1 
\hypersetup{
    colorlinks=true,
    linkcolor=blue,citecolor=blue, urlcolor=blue}


%%%%%%%%%%Glosses & related packages%%%%%%%%%%%%%%%
\usepackage[inline]{leipzig}
\usepackage{langsci-gb4e}
%\noautomath	
\makeglossaries
\glssetwidest{xxxxxxxx}
\newcommand{\Tposs}{{\Third}{\Poss}}%New fusional gloss based on predefined ones
\newleipzig{dsit}{dsit}{sitting}
\newleipzig{dgng}{dgng}{determiner going, far}
\newleipzig{coord}{coord}{coordinant}
\newleipzig{locor}{locor}{locative-orientative}
\newleipzig{disc}{disc}{discourse maker}
\newleipzig{dstd}{dstd}{determiner standing}
\newleipzig{trvz}{trvz}{transitivizer}
\newleipzig{coll}{coll}{collective}
\newleipzig{vm}{vm}{valence modifier} %Method to include new glosses
\newcommand{\Tintr}{{\Third}{\Intr}}
\newcommand{\Cl}{{\Third}{\Intr}}
\newleipzig{cl}{cl}{class}
\newleipzig{perf}{perf}{perfective}
\newleipzig{s}{s}{subject}
\title{\textbf{Herramientas digitales para el procesamiento y análisis de datos lingüísticos}}


\author{\textit{Cristian R. Juárez} \vspace{-3mm}}
\affil{Instituto Federal de Tecnología de Zürich\vspace{-3mm}}
\affil{\href{mailto:juarezcristianr@gmail.com}{juarezcristianr@gmail.com}}
\date{}



%%%%%%%%%%%%%%Document%%%%%%%%%%%%%%%%%%%%%%

\begin{document}
\maketitle

\section{Descripción}
El objetivo de este taller es presentar y utilizar una serie de herramientas digitales, \textit{Lameta}, \textit{ELAN} y \textit{FLEx}, diseñadas para la creación y el mantenimiento una base de datos lingüísticos. Vamos a trabajar sobre principios básicos para la sistematización de datos y su anotación mediante estas herramientas. Esto incluye el manejo de datos audiovisuales crudos hasta la creación de narraciones segementadas y glosadas. Mediante la integración de estas herramientas digitales, vamos a lograr una base de datos robusta para tareas de largo plazo que son esenciales en prácticas de documentación y descripción lingüística. El corpus producido mediante estas herramientas será el insumo principal para el analisis lingüístico cualitativo y cuantitativo.

\section{Cronograma de sesiones y actividades}

\subsection*{Semana 1}
\begin{itemize}
\item Lameta:
\begin{itemize}
\item Creación de la colección principal. 
\item Creación de sesiones y participantes.
\item Incluir contenido a las sesiones: título, palabras claves, descripción de sesión, etc. 
\item Exportar base de datos.
\end{itemize}
\end{itemize}

\subsection*{Semana 2}
\begin{itemize}
\item ELAN y Keyman:
\begin{itemize}
\item Instalación de Keyman y el teclado AFI.
\item Creación del archivo .eaf.
\item Cómo crear segmentaciones y anotaciones.
\item Cómo realizar búsquedas.
\item Extraer audio y subtítulos.
\item Integración con ELAN y FLEx.
\end{itemize}
\end{itemize}

\subsection*{Semanas 3}
\begin{itemize}
\item ELAN - FLEx
\begin{itemize}
\item Instalación de FLEx. 
\item Funciones básicas de FLEX. 
\item Exportar archivo ELAN a FLEx.
\item Segmentación y glosado en FLEx. 
\item Creación de entradas léxicas y glosas.
\item Cómo exportar documento de FLEx.
\end{itemize}
\end{itemize}
\subsection*{Semanas 4}
 
\begin{itemize}
\item FLEx - ELAN
\begin{itemize}
\item Revision de texto glosado y otras funciones de FLEX.
\item Cómo exportar documento de FLEx a ELAN.
\item Otros programas: IPAHelp, Phonology Assistant.
\end{itemize}
\end{itemize}


\section{Datos y programas}
Las sesiones de trabajo incluyen una breve descripción seguida de práctica aplicada a datos lenguas Guaycurues, toba y mocoví.

\subsection{Lameta}
Lameta \citep{hatton.etal2021} nos ayuda a generar metadatos para mantener un registro exaustivo de nuestra colección. Permite anotar detalles sobre el contenido de cada sesión de trabajo e incluir información relevante sobre participantes, además de muchos otros detalles que exploraremos en clase. Esta collección de sesiones forma parte de una base de datos fácil de exportar y manipular para la búsqueda de contenido específico.
\begin{itemize}
\item sitio web: \url{https://sites.google.com/site/metadatatooldiscussion/home/about-lameta}
\item instalación: \url{https://github.com/onset/lameta/releases/tag/v2.3.16-beta} 
\end{itemize}


\subsection{Eudico Languge Anotator (ELAN)}

\citet{elan} permite realizar anotaciones alineadas a audio y video. Las funcionalidades que ELAN brinda son múltiples, pero aquí solamente exploraremos las opciones de transcripción y traducción de los datos además de otras posibles integraciones con programas como Praat y su posible uso con R.
\begin{itemize}
\item sitio web: \url{https://archive.mpi.nl/tla/elan} 

\item instalación: \url{https://archive.mpi.nl/tla/elan/download}
\end{itemize}
\subsubsection{Keyman}
Es un teclado de acceso libre que permite utilizar caracteres del Alfabeto Fonético Internacional en cualquier otro programa o plataforma. Para entender cómo funciona el teclado, pueden leer más acá: \url{https://help.keyman.com/keyboard/sil_ipa/2.0.1/sil_ipa#ipa109}

\begin{itemize}
\item sitio web: \url{https://keyman.com/}
\item descarga: \url{https://keyman.com/windows/}
\item descarga del teclado AFI: \url{https://keyman.com/keyboards/h/ipa/}
\end{itemize}

\subsection{Fieldworks Language Explorer (FLEx)}
FLEx es un programa diseñado para la creación progresiva de diccionarios y análisis morfológico de textos. Nosotros vamos a utilizarlo principalmente para el análisis de de textos y veremos de qué manera podemos integrar el resultado final a ELAN. 

\begin{itemize}
\item sitio web: \url{https://software.sil.org/fieldworks/}

\item instalación: \url{https://software.sil.org/fieldworks/download/}

\item tutoriales: \url{https://software.sil.org/fieldworks/download/training-videos/}
\end{itemize}

\subsection{Otros programas}

\begin{itemize}
\item IPA Help: \url {https://software.sil.org/ipahelp2-1/}
\item Phonology Assistant: \url{https://software.sil.org/phonologyassistant/}
\end{itemize}

\bibliography{/Users/cristian/Documents/bibliotek/ZotMendlybrary.bib}
\bibliographystyle{chicago}
\end{document}